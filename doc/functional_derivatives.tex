\documentclass[11pt]{article}
\usepackage{amssymb,amsmath}
\usepackage{graphicx}
\usepackage{hyperref}
\renewcommand{\vec}[1]{\boldsymbol{#1}}
\hypersetup{
    colorlinks=true,
    linkcolor=blue,
    citecolor=red,
    filecolor=magenta,      
    urlcolor=blue
    }
\title{Functional derivatives}
\author{Eike Mueller}
\usepackage[margin=2cm]{geometry}
\begin{document}
\maketitle
In this document we consider a functional of the form $F(\alpha,u;\cdot)$ where, for a given control $\alpha\in M$, the function $u=u(\alpha)\in V$ is given by the solution of $F(\alpha,u;v) = 0$ for all $v\in V$. With $u:M\rightarrow V$ being the forward map, we are interested in back-propagating derivatives. This is described by the linear map $\mathcal{J}^*:V^* \rightarrow M^*$ which maps vectors in the dual space $V^*$ to the dual space $M^*$. As described in \cite{bouziani2023physics}, Firedrake allows the automatic construction of a differentiable PyTorch function which provides both the forward map $u=u(\alpha)$ and the backward map described by $\mathcal{J}^*_u$.

As a concrete example, let $\alpha\in M$ be a coefficient function, where we assume that $M$ is the space of piecewise constant functions on a mesh covering the domain. For some given function $f$ the solution $u=u(\alpha)\in V$ is obtained by solving
\begin{equation}
    F(\alpha,u;v) =\int_\Omega e^{\alpha}\nabla u \cdot \nabla v\;dx - \int_\Omega f v\; dx = 0 \qquad\text{for all $v\in V$}\label{eqn:F_definition}
\end{equation}
where $V$ is the space of piecewise linear functions.
%%%%%%%%%%%%%%%%%%%%%%%%%%%%%%%%%%%%%%%%%%%%%%%%%%%%%%%%%%%%%%%%%%%
\section{Preliminaries}
%%%%%%%%%%%%%%%%%%%%%%%%%%%%%%%%%%%%%%%%%%%%%%%%%%%%%%%%%%%%%%%%%%%
%%%%%%%%%%%%%%%%%%%%%%%%%%%%%%%%%%%%%%%%%%%%%%%%%%%%%%%%%%%%%%%%%%%
\subsection{Function spaces, linear maps and duals}
%%%%%%%%%%%%%%%%%%%%%%%%%%%%%%%%%%%%%%%%%%%%%%%%%%%%%%%%%%%%%%%%%%%
For two (function) spaces $V$, $W$ we write $\mathcal{N}(V,W)$ for the set of all non-linear maps from $V$ to $W$ and $\mathcal{L}(V,W)\subset \mathcal{N}(V,W)$ for the corresponding set of linear maps: for all $\ell\in\mathcal{L}(V,W)$ we have that $\ell(\alpha_1 v_1+\alpha_2 v_2) = \alpha_1\ell(v_1)+\alpha_2\ell(v_2)$. In particular, we define the dual $V^* := \mathcal{L}(V,\mathbb{R})$ of the space $V$ to be the set of all linear functionals on $V$.
%%%%%%%%%%%%%%%%%%%%%%%%%%%%%%%%%%%%%%%%%%%%%%%%%%%%%%%%%%%%%%%%%%%
\subsection{Gateux derivative}
%%%%%%%%%%%%%%%%%%%%%%%%%%%%%%%%%%%%%%%%%%%%%%%%%%%%%%%%%%%%%%%%%%%
 For a function $f\in \mathcal{N}(V,W)$ with $f(v) = w$ we  can define the Gateaux derivative in the direction $h_v$ as follows:
\begin{equation}
    \frac{df}{dv}(v;h_v) = \lim_{\tau\rightarrow 0}\frac{f(v+\tau h_v)-f(v)}{\tau}
\end{equation}
Hence, for given $v$, the derivative $\frac{df}{dv}(v;\cdot)\in \mathcal{L}(V,W)$ is a linear map from $V$ to $W$.
\paragraph{Example.} Let $\ell\in \mathcal{N}(V,\mathbb{R})$ be defined as
\begin{equation}
    \ell(u) = \int_\Omega u^2 v\; dx \qquad\text{for all $v\in V$}
\end{equation}
Then
\begin{equation}
    \begin{aligned}
    \frac{d\ell}{d u}(u;h_u) &= \lim_{\tau\rightarrow 0} \frac{1}{\tau}\left(\int_\Omega (u+\tau h_u)^2 v\; dx - \int_\Omega u^2 v\; dx\right)\\
    &= 2\int_\Omega h_u u v\; dx
    \end{aligned}
\end{equation}
and $\frac{d\ell}{d u}(u;\cdot) \in V^* = \mathcal{L}(V,\mathbb{R})$ is a linear functional.
%%%%%%%%%%%%%%%%%%%%%%%%%%%%%%%%%%%%%%%%%%%%%%%%%%%%%%%%%%%%%%%%%%%
\subsection{Adjoint operator}
%%%%%%%%%%%%%%%%%%%%%%%%%%%%%%%%%%%%%%%%%%%%%%%%%%%%%%%%%%%%%%%%%%%
Let $L\in\mathcal{L}(V,W)$ be a linear operator. Then we define the adjoint operator $L^*\in\mathcal{L}(W^*,V^*)$ as follows: for all $w'\in W^*$ we have that
$L^*(w') = v' \in V^*$ such that $v'(v) = w'(Lv)$ for all $v\in V$. In other words,
\begin{equation}
    \left(L^*(w')\right)(v) = w'(Lv)\qquad\text{for all $w'\in W^*$, $v\in V$.}
\end{equation}
Writing $\langle v',v\rangle := v'(v)$ for the duality pairing on $V$ (with a similar expression on $W$), this can also be written as $\langle L^*w',v\rangle = \langle w', Lv\rangle$.
%%%%%%%%%%%%%%%%%%%%%%%%%%%%%%%%%%%%%%%%%%%%%%%%%%%%%%%%%%%%%%%%%%%
\section{Detailed derivation}
%%%%%%%%%%%%%%%%%%%%%%%%%%%%%%%%%%%%%%%%%%%%%%%%%%%%%%%%%%%%%%%%%%%
As in \cite{bouziani2023physics}, we observe that
\begin{equation}
    \mathcal{J}^* = \left(\frac{du}{d\alpha}\right)^* \in \mathcal{L}(V^*,M^*)
\end{equation}
or more specifically:
\begin{equation}
    \langle\mathcal{J}^*w',h_\alpha \rangle  = \langle \left(\frac{du}{d\alpha}\right)^* w',h_\alpha \rangle = \langle w',\frac{du}{d\alpha}(\alpha,h_\alpha) \rangle \qquad\text{for all $w'\in V^*$, $h_\alpha\in M$}.
\end{equation}
Now consider the map $G\in\mathcal{N}(M,V^*)$ defined by
\begin{equation}
    G(\alpha)(v) = F(u(\alpha),\alpha;v) \qquad\text{for all $v\in V$.}
\end{equation} 
We have that $\frac{dG}{d\alpha}\in \mathcal{L}(M,V^*)$ and per definition of $u=u(\alpha)$ it follows that
\begin{equation}
     0 = \frac{dG}{d\alpha} = \frac{\partial F}{\partial u}\frac{du}{d\alpha} + \frac{\partial F}{\partial \alpha},
\end{equation}
where it should be noted that $\frac{\partial F}{\partial u}\in \mathcal{L}(V,V^*)$, $\frac{du}{d\alpha}\in \mathcal{L}(M,V)$ and $\frac{\partial F}{\partial \alpha}\in \mathcal{L}(M,V^*)$. Hence, we get that
\begin{equation}
    \frac{du}{d\alpha} = - \left(\frac{\partial F}{\partial u}\right)^{-1}\frac{\partial F}{\partial\alpha}
\end{equation}
With this we have that 
\begin{equation}
    \begin{aligned}
    \langle\mathcal{J}^*w',h_\alpha \rangle &= -\langle w',\left(\frac{\partial F}{\partial u}\right)^{-1}\frac{\partial F}{\partial\alpha}(\alpha,h_\alpha)\rangle\\
    &= -\langle \frac{\partial F}{\partial\alpha}(\alpha,h_\alpha), \lambda\rangle
    \end{aligned}\label{eqn:adjoint_equation}
\end{equation}
with $\left(\frac{\partial F}{\partial u}\right)^* \lambda = w'$, i.e. for each given $w'\in V^*$ we obtain $\lambda\in V$ by solving the adjoint equation
\begin{equation}
\langle \frac{\partial F}{\partial u}v,\lambda\rangle = \langle w',v\rangle \qquad\text{for all $v\in V$}.
\end{equation}
%%%%%%%%%%%%%%%%%%%%%%%%%%%%%%%%%%%%%%%%%%%%%%%%%%%%%%%%%%%%%%%%%%%
\section{Worked example}
%%%%%%%%%%%%%%%%%%%%%%%%%%%%%%%%%%%%%%%%%%%%%%%%%%%%%%%%%%%%%%%%%%%
Assume that the objective function is
\begin{equation}
    J(u,\alpha) = \int_\Omega (u-u_{\text{target}})^2\;dx + \int_\Omega \nabla\alpha\cdot \nabla\alpha\;dx
\end{equation}
where $u_{\text{target}}$ is given and $u=u(\alpha)$ satisfies \eqref{eqn:F_definition}.

We want to compute
\begin{equation}
\frac{d J}{d \alpha} \in M^*
\end{equation}
By using \eqref{eqn:adjoint_equation} with $w'=\partial J/\partial u\in V^*$ this is given by
\begin{equation}
    \begin{aligned}
\frac{d J}{d \alpha}(h_\alpha) &= \frac{\partial J}{\partial u}\frac{du}{d\alpha}(h_\alpha) + \frac{\partial J}{\partial\alpha}(h_\alpha)\\
&= -\langle\frac{\partial J}{\partial u},\left(\frac{\partial F}{\partial u}\right)^{-1} \frac{\partial F}{\partial \alpha}(h_\alpha)\rangle +\frac{\partial J}{\partial\alpha}(h_\alpha)\\
&= -\langle \frac{\partial F}{\partial \alpha}(h_\alpha),\lambda \rangle +\frac{\partial J}{\partial\alpha}(h_\alpha)\label{eqn:alpha_gradient}
    \end{aligned}
\end{equation}
for all $h_\alpha\in M$ where $\lambda\in V$ solves the adjoint equation
\begin{equation}
\langle\frac{\partial F}{\partial u}v,\lambda\rangle = \frac{\partial J}{\partial u}(v) \quad\text{for all $v\in V$}.\label{eqn:adjoint_equation_specfic}
\end{equation}
\paragraph{Step 1: Compute gradients of loss-function.} We start by computing the gradients $\partial J/\partial u\in V^*$ and $\partial J/\partial \alpha\in M^*$:
\begin{equation}
    \begin{aligned}
\frac{\partial J}{\partial u}(\alpha,u;h_u) &= \lim_{\tau\rightarrow 0}\frac{J(u+\tau h_u,\alpha)-J(u,\alpha)}{\tau}=2\int_\Omega (u-u_{\text{target}})h_u\; dx\\
\frac{\partial J}{\partial \alpha}(\alpha,u;h_\alpha) &= \lim_{\tau\rightarrow 0}\frac{J(u,\alpha+\tau h_\alpha)-J(u,\alpha)}{\tau} = 2\int_\Omega \nabla\alpha \cdot \nabla h_\alpha\;dx
    \end{aligned}
\end{equation}
\paragraph{Step 2: Solve constraint equation.} Next, we solve $F(\alpha,u;v) = 0$, i.e. for a given $\alpha\in M$ we find $u=u(\alpha)\in V$ such that
\begin{equation}
\int_\Omega e^{\alpha}\nabla u \cdot \nabla v\;dx = \int_\Omega f v\; dx\qquad\text{for all $v\in V$}.
\end{equation}
\paragraph{Step 3: Solve adjoint equation.} To derive the adjoint equation, we compute the derivative $\partial F/\partial u$. For fixed $v\in V$, this is the linear map from $V$ to $\mathbb{R}$ defined by
\begin{equation}
\frac{\partial F}{\partial u}(\alpha,u;v,h_u) = \lim_{\tau\rightarrow 0}\frac{F(\alpha,u+\tau h_u;v)-F(\alpha,u;v)}{\tau} = \int_\Omega e^\alpha \nabla h_u\cdot \nabla v\; dx
\end{equation}
Hence, the adjoint equation in \eqref{eqn:adjoint_equation_specfic} becomes
\begin{equation}
\frac{\partial F}{\partial u}(\alpha,u;\lambda,v) = \int_\Omega e^\alpha \nabla \lambda\cdot \nabla v\;dx = 2\int_\Omega (u-u_{\text{target}})v\;dx = \frac{\partial J}{\partial u}(u,\alpha;v)\quad\text{for all $v\in V$.}
\end{equation}
Solving this equation for the given $\alpha$ and $u=u(\alpha)$ gives $\lambda$.
\paragraph{Step 4: Compute $\boldsymbol{\partial F/\partial \alpha}$.} We also need the partial derivative $\partial F/\partial \alpha$. For fixed $v\in V$ this is given by
\begin{equation}
\frac{\partial F}{\partial \alpha}(h_\alpha) = \lim_{\tau\rightarrow 0}\frac{F(u,\alpha+\tau h_\alpha;v)-F(u,\alpha;v)}{\tau} = \int_{\Omega} h_\alpha e^\alpha \nabla u\cdot \nabla v\;dx.
\end{equation}
\paragraph{Step 5: Combine derivatives.} Finally, we insert everything into \eqref{eqn:alpha_gradient} to obtain
\begin{equation}
\frac{d J}{d\alpha}(h_\alpha) = -\int_\Omega h_\alpha e^{\alpha} \nabla u\cdot \nabla \lambda\;dx + 2\int_\Omega \nabla\alpha\cdot \nabla h_\alpha\;dx.
\end{equation}
%%%%%%%%%%%%%%%%%%%%%%%%%%%%%%%%%%%%%%%%%%%%%%%%%%%%%%%%%%%%%%%%%%%
\bibliographystyle{unsrt}
\bibliography{functional_derivatives}
\end{document}